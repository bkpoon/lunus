\documentstyle[12pt]{article}
\begin{document}
The following is an outline of the procedure for obtaining
three-dimensional maps of diffuse intensity:

\begin{enumerate}

\item IMAGE ORIENTATION:
\begin{enumerate}
\item Use DENZO and SCALEPACK to find the crystallographic parameters
from the oscillations.
\item Use DENZO to output the crystallographic orientation matrix for
each of the stills.
\end{enumerate}
\item IMAGE PROCESSING:
\begin{enumerate}
\item Threshold the stills to mark overflow pixels ({\tt thrshim,
proc.mode}). 
\item Window the stills to define image borders ({\tt windim,
proc.mode}). 
\item Apply a mode filter to get rid of Bragg peaks ({\tt modeim,
proc.mode}). 
\item Find the polarization of the beam by analyzing a typical 
diffraction image.  For example, use {\tt TV6} to obtain an azimuthal
intensity distribution in a thin annulus about the beam spot, and use
{\tt gnuplot} to fit the polarization.
\item Correct for the polarization effect using appropriate parameters
({\tt polarim, proc.polarim}). 
\item Correct for solid-angle normalization ({\tt normim,
proc.normim}).
\item Calculate average properties of images ({\tt avgrim, subrfim,
avsqrim, proc.avgim}).
\end{enumerate}
\item LATTICE GENERATION:
\begin{enumerate}
\item Calculate image scale factors ({\tt avgrf, proc.makeref,
proc.scale}). 
\item Generate a {\tt genlat} input file called, for example, {\tt
genlat.input} ({\tt proc.makeline, 
proc.genlat.input}). 
\item Generate the lattice ({\tt genlat, genlat.input, proc.genlat}).
\item Transform the lattice, if necessary, to correct for DENZO
orientation errors ({\tt xflt}).
\end{enumerate}
\item VISUALIZATION
\begin{enumerate}
\item {\bf EXPLORER} can be used to view the lattice immediately after
it is generated.  Run EXPLORER from the {\tt ./visualization}
directory, and open the \newline {\tt mapview.map} map, which uses
{\tt mw.float} as a template for reading lattices.  Enter the filename
of the lattice in the appropriate box, and select a threshold for the
isosurface rendering routine.  The view can be manipulated in the
Render window using the mouse.

\item {\bf Shell images} can be generated using {\tt shimlt} (reduced
image) or {\tt shim4lt} (whole image).  They can be viewed using the
{\tt xseesh} scripts.  The script {\tt proc.shimlt} can be used to
generate a sequence of shell images in increasing resolution shells.
Shell images are easiest viewed from lattices which have had their
spherically-averaged component subtracted, using {\tt avgrlt} and {\tt
subrflt}.  Otherwise, viewing thresholds have to be set by hand to
bracket the average value in the shell.
\end{enumerate}
\item ANALYSIS
\begin{enumerate}
\item The {\bf internal symmetry} of the lattice can be characterized by
calculating the difference between a lattice and its symmetry-averaged
counterpart.  The only symmetry-averaging currently implemented in
{\tt symlt} is P4$_1$.  The file {\tt script.sym} contains a
transcript of a sample session from the unix shell.

\item The techniques to
calculate the best-fit {\bf scale factor} between two lattices are
used to determine the {\bf reproducibility} of diffuse maps, and to measure
the {\it difference} between the crystals of nuclease both with and without
substrate analog.  A transcript of a session where the best scale
factor was found between two lattices is in {\tt script.scale}.  

\item Results are {\bf displayed} using {\tt gnuplot}.  The 
file {\tt script.plot} shows a transcript of a sesson with {\tt
gnuplot} to display results.
\end{enumerate}
\end{enumerate}
\end{document}

